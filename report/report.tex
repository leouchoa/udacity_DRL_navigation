% Options for packages loaded elsewhere
\PassOptionsToPackage{unicode}{hyperref}
\PassOptionsToPackage{hyphens}{url}
%
\documentclass[
]{article}
\usepackage{lmodern}
\usepackage{amssymb,amsmath}
\usepackage{ifxetex,ifluatex}
\ifnum 0\ifxetex 1\fi\ifluatex 1\fi=0 % if pdftex
  \usepackage[T1]{fontenc}
  \usepackage[utf8]{inputenc}
  \usepackage{textcomp} % provide euro and other symbols
\else % if luatex or xetex
  \usepackage{unicode-math}
  \defaultfontfeatures{Scale=MatchLowercase}
  \defaultfontfeatures[\rmfamily]{Ligatures=TeX,Scale=1}
\fi
% Use upquote if available, for straight quotes in verbatim environments
\IfFileExists{upquote.sty}{\usepackage{upquote}}{}
\IfFileExists{microtype.sty}{% use microtype if available
  \usepackage[]{microtype}
  \UseMicrotypeSet[protrusion]{basicmath} % disable protrusion for tt fonts
}{}
\makeatletter
\@ifundefined{KOMAClassName}{% if non-KOMA class
  \IfFileExists{parskip.sty}{%
    \usepackage{parskip}
  }{% else
    \setlength{\parindent}{0pt}
    \setlength{\parskip}{6pt plus 2pt minus 1pt}}
}{% if KOMA class
  \KOMAoptions{parskip=half}}
\makeatother
\usepackage{xcolor}
\IfFileExists{xurl.sty}{\usepackage{xurl}}{} % add URL line breaks if available
\IfFileExists{bookmark.sty}{\usepackage{bookmark}}{\usepackage{hyperref}}
\hypersetup{
  pdftitle={report},
  pdfauthor={Leonardo Uchoa Pedreira},
  hidelinks,
  pdfcreator={LaTeX via pandoc}}
\urlstyle{same} % disable monospaced font for URLs
\usepackage[margin=1in]{geometry}
\usepackage{graphicx,grffile}
\makeatletter
\def\maxwidth{\ifdim\Gin@nat@width>\linewidth\linewidth\else\Gin@nat@width\fi}
\def\maxheight{\ifdim\Gin@nat@height>\textheight\textheight\else\Gin@nat@height\fi}
\makeatother
% Scale images if necessary, so that they will not overflow the page
% margins by default, and it is still possible to overwrite the defaults
% using explicit options in \includegraphics[width, height, ...]{}
\setkeys{Gin}{width=\maxwidth,height=\maxheight,keepaspectratio}
% Set default figure placement to htbp
\makeatletter
\def\fps@figure{htbp}
\makeatother
\setlength{\emergencystretch}{3em} % prevent overfull lines
\providecommand{\tightlist}{%
  \setlength{\itemsep}{0pt}\setlength{\parskip}{0pt}}
\setcounter{secnumdepth}{-\maxdimen} % remove section numbering

\title{report}
\author{Leonardo Uchoa Pedreira}
\date{7/28/2021}

\begin{document}
\maketitle

{
\setcounter{tocdepth}{2}
\tableofcontents
}
\hypertarget{description}{%
\section{Description}\label{description}}

This document is a report describing the learning algorithm and details
of implementation, along with ideas for future work.

\hypertarget{algorithm-dqn-and-q-learning}{%
\section{Algorithm: DQN and
Q-Learning}\label{algorithm-dqn-and-q-learning}}

The algorithm used is called \texttt{DQN}, where a neural network is
used is to implement the \texttt{Q-Learning} Algorithm. The
\texttt{Q-Learning} algorithm attempts to estimate action-value pairs in
order to maximize the expected total reward and, therefore, to obtain
the optimal policy for the given task.

The \texttt{Q-Learning} algorithm belongs to class of value-based
methods, whose goal is to solve the
\href{https://en.wikipedia.org/wiki/Bellman_equation}{bellman equation}.
Solving the bellman equation gives us the optimal policy, given that our
environment meets certain criteria in our Markov Decision Process
setting. For \texttt{Q-Learning} in particular the equation we're trying
to solve is as follows

\[
\displaystyle Q^{new}(s_{t},a_{t})\leftarrow \underbrace {Q(s_{t},a_{t})} _{\text{old value}}+\underbrace {\alpha } _{\text{learning rate}}\cdot \overbrace {{\bigg (}\underbrace {\underbrace {r_{t}} _{\text{reward}}+\underbrace {\gamma } _{\text{discount factor}}\cdot \underbrace {\max _{a}Q(s_{t+1},a)} _{\text{estimate of optimal future value}}} _{\text{new value (temporal difference target)}}-\underbrace {Q(s_{t},a_{t})} _{\text{old value}}{\bigg )}} ^{\text{temporal difference}} 
\]

So in a given time step \texttt{t} we search for the action that
maximizes the action-value pair \(Q(s_{t+1},a)\) (and hence why this
algorithm belongs to the tabular methods class)

\end{document}
